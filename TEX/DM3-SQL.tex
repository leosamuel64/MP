\documentclass[10pt]{article}


	
    \usepackage[utf8x]{inputenc}
    \usepackage{stmaryrd}
	\usepackage[T1]{fontenc}
    \usepackage[french]{babel}
    \usepackage{minted}
    
    
    \title{DM3 - Base de données}
    \author{Léo SAMUEL}
    
	\setlength{\hoffset}{-18pt}        
\setlength{\oddsidemargin}{0pt} % Marge gauche sur pages impaires
\setlength{\evensidemargin}{9pt} % Marge gauche sur pages paires
\setlength{\marginparwidth}{54pt} % Largeur de note dans la marge
\setlength{\textwidth}{481pt} % Largeur de la zone de texte (17cm)
\setlength{\voffset}{-18pt} % Bon pour DOS
\setlength{\marginparsep}{7pt} % Séparation de la marge
\setlength{\topmargin}{0pt} % Pas de marge en haut
\setlength{\headheight}{10pt} % Haut de page
\setlength{\headsep}{10pt} % Entre le haut de page et le texte
\setlength{\footskip}{27pt} % Bas de page + séparation
\setlength{\textheight}{680pt} % Hauteur de la zone de texte (25cm)

	\newtheorem{ex}{Exemple}
	\newtheorem{prop}{Propriété}
	\newtheorem{dem}{Démonstration}
	\newtheorem{rq}{Remarque}
    \newtheorem{lem}{Lemme}
    
    






\begin{document}
    
    \maketitle

\section{Utilisation directe de la base}

Question 1 :\\

\begin{minted}[
    frame=lines,
    framesep=2mm,
    baselinestretch=1.2,
    bgcolor=white,
    fontsize=\footnotesize,
    linenos]
    {sql}
SELECT ville
FROM Postes 
WHERE region = "Québec"
\end{minted}

Question 2 :\\

\begin{minted}[
    frame=lines,
    framesep=2mm,
    baselinestretch=1.2,
    bgcolor=white,
    fontsize=\footnotesize,
    linenos]
    {sql}
SELECT SUM(nbPeaux)
FROM Achats
WHERE annee = 1871 AND espece = "lynx"
\end{minted}

Question 3 :\\

\begin{minted}[
    frame=lines,
    framesep=2mm,
    baselinestretch=1.2,
    bgcolor=white,
    fontsize=\footnotesize,
    linenos]
    {sql}
SELECT SUM(nbPeaux)
FROM Achats JOIN Postes ON Achats.idPoste = Postes.idPoste
WHERE annee = 1871 AND region = "Manitoba" AND espece = "lynx"
\end{minted}

Question 4 :\\

\begin{minted}[
    frame=lines,
    framesep=2mm,
    baselinestretch=1.2,
    bgcolor=white,
    fontsize=\footnotesize,
    linenos]
    {sql}
SELECT annee, SUM(nbPeaux) AS Quantite
from Achats
WHERE espece = "lynx"
GROUP BY annee
\end{minted}

Question 5 :\\

\begin{minted}[
    frame=lines,
    framesep=2mm,
    baselinestretch=1.2,
    bgcolor=white,
    fontsize=\footnotesize,
    linenos]
    {sql}
SELECT max(Quantite) 
FROM (	SELECT annee, SUM(nbPeaux) AS Quantite
        from Achats
        WHERE espece = "lynx"
        GROUP BY annee
	 )
\end{minted}

\newpage

Question 6 :\\

\begin{minted}[
    frame=lines,
    framesep=2mm,
    baselinestretch=1.2,
    bgcolor=white,
    fontsize=\footnotesize,
    linenos]
    {sql}
SELECT annee,SUM(QuantiteLievre) AS PeauxLievre,SUM(QuantiteLynx) AS PeauxLynx
FROM(
        SELECT annee, SUM(nbPeaux) AS QuantiteLievre,0 AS QuantiteLynx
        from Achats
        WHERE espece = "lievre"
        GROUP BY annee
        
        UNION
        
        SELECT annee, 0 AS QuantiteLievre,SUM(nbPeaux) AS QuantiteLynx
        from Achats
        WHERE espece = "lynx"
        GROUP BY annee
) group by annee    
\end{minted}

Question 7 :\\

\begin{minted}[
    frame=lines,
    framesep=2mm,
    baselinestretch=1.2,
    bgcolor=white,
    fontsize=\footnotesize,
    linenos]
    {sql}
SELECT Achats.idPoste, AVG(prix)
FROM Achats JOIN Postes on Achats.idPoste = Postes.idPoste
GROUP BY Achats.idPoste
\end{minted}

Question 8 :\\

\begin{minted}[
    frame=lines,
    framesep=2mm,
    baselinestretch=1.2,
    bgcolor=white,
    fontsize=\footnotesize,
    linenos]
    {sql}
SELECT Achats.idPoste, MIN(prix)
FROM Achats JOIN Postes on Achats.idPoste = Postes.idPoste
\end{minted}

\section{Interfaçage Python}

On suppose que la fonction \textbf{executeRequete} renvoie un tableau\footnote{ou un tableau de tableau si il y a plusieurs lignes}\\

Question 1 :\\

\begin{minted}[
    frame=lines,
    framesep=2mm,
    baselinestretch=1.2,
    bgcolor=white,
    fontsize=\footnotesize,
    linenos]
    {python}
def NombreDePeaux(année):
    t=executeRequete("""
    SELECT *
    FROM(
        SELECT annee,SUM(QuantitéLièvre) AS PeauxLièvre,SUM(QuantitéLynx) AS PeauxLynx
        FROM(
		SELECT annee, SUM(nbPeaux) AS QuantitéLièvre,0 AS QuantitéLynx
		from Achats
		WHERE espece = "lievre"
		GROUP BY annee
	  
	  UNION
	  
	  SELECT annee, 0 AS QuantitéLièvre,SUM(nbPeaux) AS QuantitéLynx
		from Achats
		WHERE espece = "lynx"
		GROUP BY annee
    ) group by annee
    WHERE annee="""+str(année)
    )
    lièvre,lynx = t[1],t[2]

    return(lièvre, lynx)
\end{minted}

Question 2 :\\

Afin d'effacer la table Achats, un pirate peut utiliser \textbf{NombreDePeaux("1871;DROP TABLE Achats")}
\newline


Question 3 :\\

\begin{minted}[
    frame=lines,
    framesep=2mm,
    baselinestretch=1.2,
    bgcolor=white,
    fontsize=\footnotesize,
    linenos]
    {python}
def anneeManquante(T):
    d=0
    f=len(T)
    trouvé=False
    while not trouvé:
        m=(d+f)//2
        if d==f:
            trouvé=True
        elif T[m]==1845+m:
            d=m+1
        elif T[m]>1845+m:
            f=m
    return T[m]-1

\end{minted}

\end{document}


