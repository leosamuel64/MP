\begin{document}
\maketitle
\section{Présentation du jeu de sudoku}
\subsection{Préliminaires}

\textbf{Question 1 :}\newline

\begin{minted}[
    frame=lines,
    framesep=2mm,
    baselinestretch=1.2,
    bgcolor=white,
    fontsize=\footnotesize,
    linenos]
    {ocaml}
let rec appartient x l =
  match l with
  | [] -> false
  | t::_ when t=x -> true
  | t::q -> appartient x q
;;
\end{minted}

\textbf{Question 2 :}\newline
\begin{minted}[
    frame=lines,
    framesep=2mm,
    baselinestretch=1.2,
    bgcolor=white,
    fontsize=\footnotesize,
    linenos]
    {ocaml}
let rec suppression x l=
  match l with
  | [] -> []
  | t::q when t=x -> suppression x q
  | t::q -> t::(suppression x q)
;;
\end{minted}

\textbf{Question 3 :}\newline

\begin{minted}[
    frame=lines,
    framesep=2mm,
    baselinestretch=1.2,
    bgcolor=white,
    fontsize=\footnotesize,
    linenos]
    {ocaml}
let ajoute x l=
  let rec aux x l flag=
    match l with
    | [] when not flag -> [x]
    | [] -> []
    | t::q when t=x -> t::(aux x q true)
    | t::q -> t::(aux x q flag)
  in aux x l false
;;
\end{minted}

\textbf{Question 4 :}\newline

\begin{minted}[
    frame=lines,
    framesep=2mm,
    baselinestretch=1.2,
    bgcolor=white,
    fontsize=\footnotesize,
    linenos]
    {ocaml}
let rec indice (b,r)=
  let i = (b mod 3)*3 + (r mod 3) in
  let j = (b/3)*3 + r/3 in
  (i,j)
;;
\end{minted}

\section{Codage de la formule initiale}
\subsection{Formule logique décrivant la règle du jeu}

    
\textbf{Question 1 :} \newline

(a) :\newline 

Soit $(i,j) \in ⟦0,8⟧^2, \lor_{k=1}^9 x_{(i,j)}^k $ \newline 

(b) :\newline

On a : $\forall (i,j) \in ⟦0,8⟧^2, \exists k \in ⟦1,9⟧, x{^k_{(i,j)}}$\newline 

(c) :\newline

On a : $(K_1)\equiv \land_{i=0}^8(\land_{j=0}^8(\lor_{k=1}^9 x{_{(i,j)}^k}))$\newline 

(d) :\newline 

Il y a $8 \cdot 8 = 81$ clauses\newline 

(e) :\newline

\begin{minted}[
    frame=lines,
    framesep=2mm,
    baselinestretch=1.2,
    bgcolor=white,
    fontsize=\footnotesize,
    linenos]
    {ocaml}
let case1 () =
  let res = ref [] in
  for i=0 to 8 do
    for j=0 to 8 do
      let temp = ref [] in
      for k=1 to 9 do
        temp:= (X(i,j,k)) :: !temp;
      done;
    res:= !temp :: !res;
    done;
  done;
  !res
;;
\end{minted}

\textbf{Question 2 :}\newline

On a : $\forall i \in ⟦0,8⟧,\forall k \in ⟦1,9⟧, \exists j \in ⟦0,8⟧, x{^k_{(i,j)}}$ \newline
On obtient alors la clause  $L_1\equiv \land_{i=0}^8(\land_{k=1}^9(\lor_{j=0}^8 x{_{(i,j)}^k}))$\newline

\textbf{Question 3 :}\newline

(a) :\newline

On a $(C_1) \equiv \land_{j=0}^8(\land_{k=1}^9(\lor_{i=0}^8 x{_{(i,j)}^k}))$\newline
Et $(B_1) \equiv \land_{b=0}^8(\land_{k=1}^9(\lor_{r=0}^8 x{_{\text{indice}(b,r)}^k}))$\newline

\newpage
(b) :\newline

\begin{minted}[
    frame=lines,
    framesep=2mm,
    baselinestretch=1.2,
    bgcolor=white,
    fontsize=\footnotesize,
    linenos]
    {ocaml}
let bloc1 ()=
  let res = ref [] in
  for b=0 to 8 do
    for k=1 to 9 do
      let temp = ref [] in
      for r=0 to 8 do
        let i,j = indice (b,r) in
        temp:= X(i,j,k) :: !temp;
      done;
      res:= !temp :: !res
    done;
  done;
  !res
;;
\end{minted}

\textbf{Question 4 :}\newline

(a) :\newline

La ligne $i$ ne contient pas deux fois la valeur $k$ lorsque :
$\forall j_1,j_2 \in ⟦0,8⟧, j_1< j_2 \Rightarrow (\lnot x{_{(i,j_1)}^k} \lor \lnot x{_{(i,j_2)}^k})$ \newline

On obtient alors $\land_{b=0}^8 \land_{k=1}^9(\lnot x{_{(i,j_1)}^k} \lor \lnot x{_{(i,j_2)}^k})$\newline

(b) :\newline

On a $\forall i \in ⟦0,8⟧,\forall k \in ⟦1,9⟧,\forall j_1,j_2 \in ⟦0,8⟧, j_1 \not= j_2 \Rightarrow (\lnot x{_{(i,j_1)}^k} \lor \lnot x{_{(i,j_2)}^k})$\newline

D'où $(L_2) \equiv \land_{i=0}^8\land_{k=1}^9\land_{j_1=0}^8\land_{j_2=0}^8(\lnot x{_{(i,j_1)}^k} \lor \lnot x{_{(i,j_2)}^k})$\newline

(c) :\newline

Il y a $9\cdot 9\cdot \frac{9\cdot 8}{2}=2916$ clauses.\newline

(d) :\newline

\begin{minted}[
    frame=lines,
    framesep=2mm,
    baselinestretch=1.2,
    bgcolor=white,
    fontsize=\footnotesize,
    linenos]
    {ocaml}
let ligne2 ()=
  let res = ref [] in
  for i=0 to 8 do
    for k=1 to 9 do
      for j1=0 to 7 do
        for j2=j1 to 8 do
          res:= [NonX(i,j1,k);NonX(i,j2,k)] :: !res;
        done;
      done;
    done;
  done;
  !res
;;
\end{minted}

\newpage
\textbf{Question 5 :}\newline

On a \newline

$(C_2) & \equiv & \land_{j=0}^8\land_{k=1}^9\land_{i_1=0}^7\land_{i_2=i_1+1}^8(\lnot x{_{(i_1,j)}^k} \lor \lnot x{_{(i_2,j)}^k}) $\newline
$(K_2) & \equiv & \land_{i=0}^8\land_{j=0}^8\land_{k_1=1}^8\land_{k_2=k_1+1}^9(\lnot x{_{(i,j)}^k_1} \lor \lnot x{_{(i,j)}^k_2}) $\newline
$(B_2) & \equiv & \land_{b=0}^8\land_{k=1}^9\land_{r_1=0}^7\land_{r_2=r_1+1}^8(\lnot x{_{\text{indice}(b,r_1)}^k} \lor \lnot x{_{\text{indice}(b,r_2)}^k})$ \newline

\subsection{Formule logique décrivant la grille initiale}

\textbf{Question 1 :} \newline

\begin{minted}[
    frame=lines,
    framesep=2mm,
    baselinestretch=1.2,
    bgcolor=white,
    fontsize=\footnotesize,
    linenos]
    {ocaml}
let donnees t=
  let res = ref [] in
  for i=0 to 8 do
    for j=0 to 8 do
      match t.(i).(j) with
      | 0 -> ()
      | k ->  for q=1 to 9 do
                if q=k then 
                  res:= [X(i,j,q)] :: !res
                else
                  res:= [NonX(i,j,q)]::!res
              done;
    done;
  done;
;;
\end{minted}

\textbf{Question 2 :} \newline

(a) :\newline

On a $b=3\left \lfloor{\frac{i}{3}}\right \rfloor+\left \lfloor{\frac{j}{3}}\right \rfloor$ \newline

(b) :\newline

\begin{minted}[
    frame=lines,
    framesep=2mm,
    baselinestretch=1.2,
    bgcolor=white,
    fontsize=\footnotesize,
    linenos]
    {ocaml}
let interdites_ij t i j=
  let res = ref [] in
  let b = 3*(i/3)+(j/3) in
  for l=0 to 8 do
    if t.(l).(j) <> 0 then
      res := ajoute (NonX(i,j,t.(l).(j))) !res;
    if t.(i).(l) <> 0 then
      res := ajoute (NonX(i,j,t.(i).(l))) !res;
    let i1,j1 = indice(b,l) in
    if t.(i1).(j1) <> 0 then
      res := ajoute (NonX(i,j,t.(i1).(j1))) !res;
  done;
!res ;;
\end{minted}
\newpage
(c) :\newline

\begin{minted}[
    frame=lines,
    framesep=2mm,
    baselinestretch=1.2,
    bgcolor=white,
    fontsize=\footnotesize,
    linenos]
    {ocaml}
let interdites t =
  let res = ref [] in
  for i = 0 to 8 do 
    for j = 0 to 8 do
      if t.(i).(j)=0 then
        res := (interdites_ij t i j) :: !res;
    done; 
  done;
!res;;
\end{minted}

\textbf{Question 3 :}\newline

Chaque case remplie est associé à 9 clauses. De plus, chaque case non remplie est associé à 8 clauses.\newline
On notant $r$ le nomvre de cases remplis dans la grille initiale.\newline

Alors on a $9r+8(81-r)$ clauses. Dans le pire des cas, la grille est remplis et $r=81$.\newline
Dans le pire des cas, il y a $9^3=729$ clauses.\newline

\section{Résolution}
\subsection{Propagation unitaire}

\textbf{Question 1 :}\newline

Depuis une grille initiale, on obtient une unique grille finale. Alors il existe une unique solution. D'où, il y a une seule valuation satisfaisant $F_{\text{initiale}}$. \newline

\textbf{Question 2 :}\newline

Il y a $729$ clauses alors la table de vérité contient $2^{729}\approx 10^{219}$ lignes.\newline

\textbf{Question 4 :}\newline

On a $x{_{(0,0)}^1}\land(x{_{(2,2)}^4} \lor x{_{(3,6)}^6} \lor x{_{(7,7)}^7}) \land (\lnot x{_{(0,0)}^1} \lor \lnot x{_{(3,6)}^6})$ \newline
Puis $(x{_{(2,2)}^4} \lor x{_{(3,6)}^6} \lor x{_{(7,7)}^7}) \land (\lnot x{_{(3,6)}^6})$\newline
Finalement $(x{_{(2,2)}^4}  \lor x{_{(7,7)}^7})$\newline

\textbf{Question 5 :}\newline

(a) :\newline

Les possibilités sont $1$,$2$,$4$ et $7$. Or il y a un $7$ présent dans le bloc $0$ à la ligne $2$ et dans le bloc $1$ à la ligne $1$ alors la seule possibilitée pour placer un $7$ dans le bloc $2$ est sur la ligne $0$ dans la case libre. \newline
\newpage
\textbf{Question 6 :}\newline

\begin{minted}[
    frame=lines,
    framesep=2mm,
    baselinestretch=1.2,
    bgcolor=white,
    fontsize=\footnotesize,
    linenos]
    {ocaml}
let rec nouveau_lit_isole f=
  match f with
  | [] -> X(-1,-1,-1)
  | [c]::q -> c
  | _::q -> nouveau_lit_isole q
;;
\end{minted}

\textbf{Question 7 :}\newline

\begin{minted}[
    frame=lines,
    framesep=2mm,
    baselinestretch=1.2,
    bgcolor=white,
    fontsize=\footnotesize,
    linenos]
    {ocaml}
let non x=
  match x with 
  | X(i,j,k) -> NonX(i,j,k)
  | NonX(i,j,k) -> X(i,j,k)
;;

let rec simplification l f=
  match f with
  | [] -> []
  | t::q when appartient l t -> simplification l q
  | t::q -> suppression (non l) t :: simplification l q
;;
\end{minted}

\textbf{Question 8 :}\newline

\begin{minted}[
    frame=lines,
    framesep=2mm,
    baselinestretch=1.2,
    bgcolor=white,
    fontsize=\footnotesize,
    linenos]
    {ocaml}
let rec propagation t f=
  match nouveau_lit_isole f with
  | X(-1,-1,-1) -> f
  | X(i,j,k) -> t.(i).(j) <-k; 
                propagation t (simplification (X(i,j,k)) f)
  | l -> propagation t (simplification l f)
;;
\end{minted}

\newpage
\subsection{Règle du littéral infructueux}

\textbf{Question 2 :}\newline

\begin{minted}[
    frame=lines,
    framesep=2mm,
    baselinestretch=1.2,
    bgcolor=white,
    fontsize=\footnotesize,
    linenos]
    {ocaml}
let variables f = 
  let res = ref [] in
  let rec ajouter_clause c = 
    match c with
    | [] -> ()
    | X(i,j,k)::q ->  res := ajoute (X(i,j,k)) !res; 
                      ajouter_clause q
    | NonX(i,j,k)::q -> res := ajoute (X(i,j,k)) !res; 
                        ajouter_clause q in
  let rec ajouter_formule f = 
    match f with
    | [] -> ()
    | t::q -> ajouter_clause t; 
                  ajouter_formule q
  in ajouter_formule f;
  !res
;;
\end{minted}

\textbf{Question 3 :}\newline

\begin{minted}[
    frame=lines,
    framesep=2mm,
    baselinestretch=1.2,
    bgcolor=white,
    fontsize=\footnotesize,
    linenos]
    {ocaml}
let copie_matrice m=
  let n,p=Array.length m,Array.length m.(0) in
  let nvmat = Array.make_matrix n p 0 in
  for i=0 to n do
    for j=0 to p do
      nvmat.(i).(j) <- m.(i).(j);
    done;
  done;
  nvmat
;;

let deduction t x f = 
  let t1 = copie_matrice t in 
  match x with
  | X(i,j,k) when propagation t1 ([NonX(i,j,k)] :: f) = [[]] -> 1
  | _ -> let t2 = copie_matrice t 
  in
  if propagation t2 ([x] :: f) = [[]] then 
    -1 
  else 
    0
;;
\end{minted}




















\end{document}