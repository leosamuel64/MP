\begin{document}

    \section{Exemples introductifs}
    Ecrivons un programme pour chercher le mot "info" dans un texte. L'algo naif à déjà été vu en MPSI.
    Il a une complexité en $O(|\text{texte}|\cdot |\text{mot|})$. \\

    Si le teste échoue à la 4^{ième} lettre, on peut reprendre la recherche à la 4_{ième} lettre 
    et pas de 0 \\

    Voici une méthode plus efficace : \\
    \begin{itemize}
        \item Lire une seul fois le texte et garder en mémoire où on en est du mot
    \end{itemize}

    Pour ce faire, utilisons le graphe suivant : \\
    [Image graphe]\\

    Mode d'emploi :
    \begin{itemize}
        \item Partir du sommet indiqué par une flèche sans sommet de départ
        \item lire le texte lettre après lettre et suivre les flèches
        \item A la fin de la lecture, si on est au sommet 4, c'est que le texte contenait "info"
    \end{itemize}

    L'exemple ci dessus est simple car "info" n'a pas de lettre en double. Voyons comment appliquer la méthode au mot "infini".\\

    [Image Graphe]\\

    Les graphes utilisé sont des "automates". L'état signalé par une flèche est dit "initial". Le ou les 
    états signalés par 0 sont dit "acceptant" ou "terminaux"

    \section{Programmation d'un automate fini déterministe}

On définie le type suivant :

\begin{minted}[
    frame=lines,
    framesep=2mm,
    baselinestretch=1.2,
    bgcolor=white,
    fontsize=\footnotesize,
    linenos]
    {ocaml}
type automate = { initial : int; 
                  finals : int list; 
                  transitions : (int*char) list array}
                ;;
    \end{minted}








    
\end{document}




