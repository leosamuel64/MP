\begin{document}
	\maketitle

    \section{Exemples introductifs}
    Ecrivons un programme pour chercher le mot "info" dans un texte. L'algo naif à déjà été vu en MPSI.
    Il a une complexité en $O(|\text{texte}|\cdot |\text{mot|})$. \\

    Si le teste échoue à la 4^{ième} lettre, on peut reprendre la recherche à la 4_{ième} lettre 
    et pas de 0 \\

    Voici une méthode plus efficace : \\
    \begin{itemize}
        \item Lire une seul fois le texte et garder en mémoire où on en est du mot
    \end{itemize}

    Pour ce faire, utilisons le graphe suivant : \\
    [Image graphe]\\

    Mode d'emploi :
    \begin{itemize}
        \item Partir du sommet indiqué par une flèche sans sommet de départ
        \item lire le texte lettre après lettre et suivre les flèches
        \item A la fin de la lecture, si on est au sommet 4, c'est que le texte contenait "info"
    \end{itemize}

    L'exemple ci dessus est simple car "info" n'a pas de lettre en double. Voyons comment appliquer la méthode au mot "infini".\\

    [Image Graphe]\\

    Les graphes utilisé sont des "automates". L'état signalé par une flèche est dit "initial". Le ou les 
    états signalés par 0 sont dit "acceptant" ou "terminaux"

    \section{Programmation d'un automate fini déterministe}

On définie le type suivant :

\begin{minted}[
    frame=lines,
    framesep=2mm,
    baselinestretch=1.2,
    bgcolor=white,
    fontsize=\footnotesize,
    linenos]
    {ocaml}
type automate = { initial : int; 
                  finals : int list; 
                  transitions : (int*char) list array}
                ;;
\end{minted}

% Programmation de Infini

\section{Vocabulaire sur les languages}

Soit $\Sigma$ un ensemble qu'on appelle "alphabet"\\

\begin{itemize}
    \item Un élément de $\Sigma$ s'appelle une lettre
    \item Une suite finie de lettres s'appelle un mot
    \item Le mot vie est noté $\epsilon$
    \item La concatenation des mots est noté $\cdot$
    \item On note $\Sigma*$ l'ensemble des mots sur $\Sigma$
    \item $\forall u \in \Sigma*, |u|$ est la longueur de u, son nombre de lettre
    \item On note $\forall n \in \mathbb{N}, \Sigma^n$ l'ensemble des mots de $n$ lettres
    \item On identifie les mot de une lettre avec les lettres cad $\Sigma = \Sigma^1$
    \item Un ensemble de mot s'appelle un langage
\end{itemize}

\begin{prop} $\cdot$
    \begin{itemize}
        \item Associatif : $\forall u,v,w \in \Sigma, (u\code v)\cdot w = u\code (v\cdot w)$
        \item Neutre : $\epsilon$
        \item Pas Commutatif dans la plupart des cas
        \item $\forall u,v \in \Sigma, |u \cdot v| = |u| + |v|$
        \item $\forall u,v \in \Sigma*, u=v <=> |u|=|v| \land \forall i \in [[0,|u|[[, u_i=v_i$
    \end{itemize}
\end{prop}

\begin{defi}
    Soit $u,v \in \Sigma*,$\\
    \begin{itemize}
        \item On dit que u est un préfixe de $v$ lorsque $\exists w \in \Sigma*, v=u\cdot w$
        \item On dit que u est un suffixe de $v$ lorsque $\exists w \in \Sigma*, v=w\cdot u$
    \end{itemize}
\end{defi}

\section{Définition d'un automate (fini déterministe)}

\begin{defi}
    Un automate fini déterministe complet sur $\Sigma$ est un quadruplet formé de
    \begin{itemize}
        \item Un ensemble $\mathcal{Q}$ appelé ensemble des états
        \item Un état paticulier $i \in \mathcal{Q}$ appelé l'état initial
        \item Un ensemble d'états $\mathcal{F} \inset \mathcal{Q}$ appelé ensemble des états finals
        \item Une fonction $\delta : \mathcal{Q}*\Sigma \rightarrow \mathcal{Q}$ appelé fonction de transitions
    \end{itemize}
\end{defi}

\begin{rmq}
    Dans le type Ocaml, on utiise un tableau de liste d'association pour enregistrer les transition
\end{rmq}

\begin{defi}
    Soit $(\mathcal{Q},i,\mathcal{F},\delta)$ un AFDC. On définie sa fonction de transition par récurrence, ainsi:
    $\delta* : \mathcal{Q}*\Sigma^* \rightarrow \mathcal{Q}$ vérifie : \\
    \begin{itemize}
        \item $\forall q \in \mathcal{Q}, \delta^*(q,\epsilon)=q$
        \item $\forall q \in \mathcal{Q}, \forall x \in \Sigma, \forall m \in \Sigma^*,\delta^*(q,mx)=\delta(\delta^*(q,m),x)$
    \end{itemize}
\end{defi}

\begin{prop}
    Soit $(\mathcal{Q},i,\mathcal{F},\delta)$ un AFDC.\\
    $\forall q \in \mathcal{Q}, \forall(m_1,m_2)\in \Sigma^2, \delta^*(q,m_1m_2)=\delta^*(\delta^*(q,m_1),m_2)$
\end{prop}

\begin{dem}
    (Récurrence sur la longueur de $m_2$)\\
    On fixe un $m_1 \in \Sigma^*$\\
    $\forall n \in \mathbb{N}, P_n$:"$\forall m_2 \in \Sigma^*,\delta^*(q,m_1m_2)= \delta^*(\delta^*(q,m_1),m_2)$"

    Initialisation : Soit $m_2 \in \Sigma^0, \delta^*(\delta^*(q,m_1),\epsilon) = \delta^*(q,m_1)$
    Hérédité : Soit $n \in \mathbb{N}$. Supposons $P_n$.\\
    Soit $m_2\in \Sigma^{n+1}$\\
    Soit $n_2\in \Sigma^{n}, x \in \Sigma, m_2=n_2x$\\
    $\delta^*(\delta^*(q,m_1),m_2)$\\
    $=\delta^*(\delta^*(q,m_1),n_2x)$\\
    $=\delta(\delta^*(\delta^*(q,m_1),n_2),x)$\\
    $=\delta(\delta^*(q,m_1n_2),x)$\\
    $=\delta(q,m_1m_2x)$\\

    $\forall m_2 \in \Sigma^*,\delta^*(q,m_1m_2)= \delta^*(\delta^*(q,m_1),m_2)$\\
\end{dem}

\begin{rmq}
    On note $q.m$ pour $\delta^*(q,m)$
    Alors $\forall q \in \mathcal{Q},\forall m_1,m_2 \in \Sigma, (q.m_1).m_2=q.(m_1\cdot m_2)$
\end{rmq}

\begin{defi}
    Language reconnu par un automate\\
    Soit $(\mathcal{Q},i,\mathcal{F},\delta)$ un AFDC que l'on note $\mathcal{A}$.\\
    \begin{itemize}
        \item $\forall m \in \Sigma^*$, on dit que m est reconnu par $\mathcal{A}$ lorsque $\delta^*(i,m) \in \mathcal{F}$
        \item On appelle langage reconnu par $\mathcal{A}$ l'ensemble des mots reconnu par $\mathcal{A}$
    \end{itemize}
\end{defi}

% Finir l'exo 10

\section{Opérations sur les langages}

On définit quelques opérations sur les langages
On connait deja $cup et inter, \ $. On notera plus tard $+$ à la place de $cup$\\

\begin{defi} concatenation de language
    Soient $L_1$ et $L_2$ deux langage. On pose $L_1L_2=\{m_1m_2;m_1 \in L_1, m_2 \in l_2\}$
\end{defi}

\begin{ex}
    $\{ga,bu}\cdot {zo,meu}={gazo,gameu,buzo,bumeu}$    
\end{ex}

\begin{defi} Point et étoile
    \begin{itemize}
        \item $forall langage L \forall n \in \mathbb N, L^n=L.L.L...L$
        \item $L^*=\cup(n\in\mathbb N) L^n$
    \end{itemize}
\end{defi}

\begin{prop} de $\cdot$\\
    \begin{itemize}
        \item associatif
        \item neutre $\{\epsilon\}$
        \item Distributif sur $\cup$ : $\forall L,M,N$ trois langages, $L\cdot (M \cup N) = L\cdot M \cup L\cdot N$ et $(L\cup M) \cdot N) = M\cdot L \cup N\cdot L$
    \end{itemize}
\end{prop}

\begin{dem}
Premiere égalité : Soit $m\in L\cdot M \cup L \cdot N$\\
Traitons le cas $m \in L\cdot M$\\
Donc $\exists x \in L, y\in M,m=x\cdot y$
Donc $l\in L\cdot(M\cup N)$

Soit $m \in L\cdot(M\cup N)$\\
Donc $\exists x \in L,\exists y \in M \cup N, m=x\cdot y$ alors $m \in L\cdot M \cup L\cdot N$
\end{dem}

\begin{rmq}
    Dans ce chapitre, on note parfois $+$ au lieu de $\cup$. De plus, le neutre pour $+$ sera $\emptyset$
\end{rmq}

\begin{prop}
    \begin{itemize}
        \item $\forall \text{ langages } L, (L^*)^*=L^*$
        \item $\forall L,M\in \mathcal{P}(\Sigma^*), L\subset M \Rightarrow L^*\subset M^*$. On dit que $*$ est croissante
    \end{itemize}
\end{prop}

\begin{dem}
    Supposons $L\subset U$\\
    Soit $m\in L^*$, $\exists p \in \mathbb{N}, (l_1,\dots,l_p)\in L^p$ tq
    $m=l_1l_2\dots l_p$\\

    Or $\forall i \in [[1,p]], l_i \in L \subset M$
    Donc $m \in M^*$
\end{dem}
    
\begin{defi} Langage régulier\\
    Un langage est dit régulier lorsqu'il peut être décrit à l'aide d'un nombre fini d'opérations :
    \begin{itemize}
        \item $+$ ou $\cup$
        \item $\cdot$
        \item $*$
        \item $\emptyset$
        \item $\epsilon$
        \item Les singletons $\{x\}$ pour $x \in \Sigma$ 
    \end{itemize}
\end{defi}

Une telle formule est appelé une expression régulière (cf suite du cours)\\

\begin{rmq}
    Il n'y à pas de $\cap$ ni de complémentaire dans cette définition
\end{rmq}

\begin{ex}
    \begin{itemize}
        \item Ensemble des texte contenant le mot info : $\Sigma^*\cdot \text{info}\cdot \Sigma^*$
        \item Soit $\Sigma = \{a,b\}$. Ensemble des mots contenant deux a: $b^*ab^*ab^*$
    \end{itemize}
\end{ex}

\begin{rmq}
$\forall x \in \Sigma$, on notera $x$ au lieu de $\{x\}$
\end{rmq}

\section{Automate incomplet}

\subsection{Définition}

La définition d'un AFD incomplet est la même que celle d'un AFD complet sauf que la fonction de transition
$\delta$ n'est pas définie sur $\mathcal{Q}*\Sigma$ \\

Un couple $(q,x) \in \mathcal{Q}*\Sigma$ où $\delta$ n'est pas définie s'appelle un blocage\\

La fonction de transition étendue $\delta^*$  est alors défine seulement sur une partie de $\mathcal{Q}*\Sigma$\\
Un mot $m\in \Sigma^*$ est reconnu lorsque
\beginmatrix
$(i,m)\in D_{\delta^*}$
$\delta*(i,m)\in \mathcal{F}$

\begin{exemple}
    AFD non complet pour reconnaitre l'ensemble des numéros de téléphone 
    \begin{itemize}
        \item Si on lit autre chose qu'un chiffre, c'est un blocage
        \item Si on lit plus de 10 chiffre, c'est encore un blocage
    \end{itemize}
\end{exemple}

Un automate incomplet est alors
\begin{itemize}
    \item Plus pratique à dessiner
    \item Plus clair
    \item Moins gourmant en mémoire et en temps : Les listes d'association sont plus courte et le programme peut s'arreter avant la fin
\end{itemize}

\subsection{Programmation}

\begin{minted}[
	frame=lines,
	framesep=2mm,
	baselinestretch=1.2,
	bgcolor=white,
	fontsize=\footnotesize,
	linenos]
	{ocaml}

exception Blocage;;

let delta2 i x a=
  (*  a : automate
      x : une lettre
      i : un état de a
  *)
  let rec aux = function
    | [] -> raise Blocage
    | (lettre, etat)::_ when lettre = x -> etat 
    | _::q -> aux q
    in
    aux (a.transitions.(i))
;;

let delta_etoile2 i m a =
  (* Cette fois m est une chaine de caracteres.
      Renvoie l'etat atteint apres lecture de toutes les lettres de m *)
  
  let rec boucle q k =
      (* k : prochaine lettre de m à lire
          q : etat actuel *)
          if k = String.length m then
              q
          else
              boucle (delta2 q m.[k] a ) (k+1)
  in 
  boucle i 0
;;

let reconnu2 m a =
  try
    List.mem (delta_etoile2 a.initial m a) a.finals
  with
    | Blocage -> false
;;

let completed a alphabet=
  let n = Array.length a.transitions in
  let p = n in
  let nv_trans = Array.make (p+1) (tout_vers p alphabet) in
  for i=0 to n-1 do
    nv_trans.(i) <- (a.transitions.(i)@ nv_trans.(i));
  done;
  {
    initial = a.initial;
    finals = a.finals;
    transitions=nv_trans
  }
;;
\end{minted}

\subsection{Complétion}

Dans certain cas, on a besoin de completer un automate incomplet.

\begin{thr}
    Soit $(\mathcal{Q},i,\mathcal{F},\delta)$ un AFD qu'on note $\mathcal{A}$

On définit un nouvel automate $\mathcal{A}'$ ainsi
\begin{itemize}
    \item Soit $p$ un nouvel état et $\mathcal{Q}'=\mathcal{Q}\cup\{p\}$
    \item On garde $i$ et $\mathcal{F}$
    \item On garde les transitions de $\mathcal{A}$
    \item On rajoute $\forall (q,x)\in \mathcal{Q}*\Sigma$ blocage dans $\mathcal{A}$ $q$->$p$ et $p$->$p$
\end{itemize}

Alors $\mathcal{A}'$ est complet et il reconnait le même langage que $\mathcal{A}$

\end{thr}

\begin{dem}
    \\
		[$\Leftarrow$] Soit $m\in\mathcal{L}(\mathcal{A})$. Soit $c$ le chemin de $\mathcal{A}$ étiqueté par
    $m$. Ce chemin existe aussi dans $\mathcal{A}'$. Donc $m$ est reconnu par $\mathcal{A}'$

    [$\Rightarrow$] Soit $m\in\mathcal{L}(\mathcal{A}')$Soit $c$ un chemin dans $\mathcal{A}'$ étiqueté par $m$, de
    $i$ à un certain $q_f$.\\
    
    $c$ ne passe pas par $p$ sans quoi il finirait à $p$ or $q_f\not= p$ car $p\not\in \mathcal{F}$\\
    Donc toutes les transitions emprunté par $c$ existent dans $\mathcal{A}$\\
    Donc $c$ est un chemin de $\mathcal{A}$
\end{dem}

\begin{cor}
    L'ensemble des language reconnaissable par un AFD complet est le même que l'ensemble des language reconnaissable par un AFD pas forcément complet
\end{cor}

\section{Automate émondé}

But : "alleger" un automate en suprimant des états inutiles

\subsection{Définitions}

\begin{definition}
    Soit $(\mathcal{Q},i,\mathcal{F},\delta)$ un AFD qu'on note $\mathcal{A}$\\
    \begin{itemize}
        \item $\forall q \in \mathcal{Q}, q$ est dit accessible lorsque $\exists m \in \Sigma^*$ tq $\delta^*(i,m)=q$
        \item $\forall q \in \mathcal{Q}, q$ est dit co-accessible lorsque $\exists m \in \Sigma^*$ tq $\delta^*(i,m) \in \mathcal{F}$
        \item $\mathcal{A}$ est dit "émondé" lorsque tous ses états sont accessible et co-accessible
    \end{itemize}
\end{definition}

On pourrait qualifier d'inutile tout état non accessible ou non co-accessible\\

\subsection{Emondage}

\begin{proposition}
    Soit $(\mathcal{Q},i,\mathcal{F},\delta)$ un AFD qu'on note $\mathcal{A}$\\
    Soit $\mathcal{Q}_é$ l'ensemble des états accessibles et co-accessibles\\
    Soit $\delat_é$ la restriction de $\delta$ à $\mathcal{Q}_é$\\

    Soit $(\mathcal{Q}_é,i,\mathcal{F}\cap \mathcal{Q}_é,\delta_é)$ un AFD qu'on note $\mathcal{A}_é$\\
    Alors $\mathcal{A}_é$ est émondé et $\mathcal{L}(\mathcal{A})=\mathcal{L}(\mathcal{A}_é)$
\end{proposition}

Ainsi quand on a obtenu un AFD qui reconnait un certain langage, on a presque toujours intérêt à l'émonder
pour obtenir un automate plus simple reconnaissant le même langage.\\

\begin{dem}
    - Montrons que $\mathcal{A}_é$ est émondé. Cad montrons que les états de $\mathcal{A}_é$ sont accessibles et co-accessible
    dans $\mathcal{A}_é$ (sachant qu'ils le sont dans $\mathcal{A}$)\\
    Soit $q\in \mathcal{Q}_é$. Donc $q$ est accessible et co-accessible\\
    Donc $\exists \gamma$ chemin dans $\mathcal{A}$ qui part de $i$ et arrive à un état de $\mathcal{F}$.\\
    Tous les états le long de $\gamma$ sont accessibles et co-accessibles donc sont encore dans $\mathcal{Q}_é$
    Donc $\gamma \subset \mathcal{Q}_é$. Donc $q$ est accessible et co-accessible dans $\mathcal{A}_é$
    Donc $\mathcal{A}_é$ est émondé

    - Montrons que $\mathcal{L}(\mathcal{A})=\mathcal{L}(\mathcal{A}_é)$\\
    Soit $m\in =\mathcal{L}(\mathcal{A}_é)$ Soit $\gamma$ un chemin acceptant dans $\mathcal{A}_é$ étiqueté par $m$. Alors $\gamma$ est aussi un chemin acceptant dans $\mathcal{A}$\\
    Donc $m \in \mathcal{L}(\mathcal{A})$\\

    Soit $m \in =\mathcal{L}(\mathcal{A})$. Soit $\gamma$ un chemin acceptant dans $\mathcal{A}$ étiqueté par $m$\\
    Avec le même raisonnement, $m \in \mathcal{L}(\mathcal{A}_é)$
\end{dem}

\subsection{Programmation}

\begin{itemize}
    \item Trouver les sommets accessibles $\rightarrow$ il suffit delancer un parcours de graphe depuis le sommet initial
    \item Trouver les sommets co-accessibles \begin{itemize}
        \item Créer un tableau coaccessible, initialement rempli de false
        \item rajouter en argument le chemin parcouru pour arriver au sommet actuel
    \end{itemize}
\end{itemize}

\begin{minted}[
    frame=lines,
    framesep=2mm,
    baselinestretch=1.2,
    bgcolor=white,
    fontsize=\footnotesize,
    linenos]
    {ocaml}

let rec accessible trans i=
	let n = Array.length trans in
	let deja_vu = Array.make n false in
	let rec visite_sommet s=
		if deja_vu.(s) then []
		else (
			deja_vu.(s) <- true;
			s::visite_voisins trans.(s)
		)
	and visite_voisins = function
		| [] -> []
		| (_,q)::suite -> (visite_sommet q) @ (visite_voisins suite)
	in
		visite_sommet i
;;

let etats_utiles a =
	let trans = a.transitions in
	let n = Array.length trans in
	let deja_vu = Array.make n false in
	let coaccessible = Array.make n false in
		
		let rec visite_sommet s chemin_parcouru=
			if deja_vu.(s) then []
			else (
				deja_vu.(s) <- true;
				if List.mem s (a.finals) then
					List.iter (fun q -> coaccessible.(q)<- true) 
										(s::chemin_parcouru);
				s::visite_voisins (s::chemin_parcouru) trans.(s)
			)
		and visite_voisins chemin_parcouru = function
			| [] -> []
			| (_,q)::suite -> (visite_sommet q chemin_parcouru) @ (visite_voisins chemin_parcouru suite)
		in
			List.filter (fun q -> coaccessible.(q)) 
									(visite_sommet a.initial [])
;;
	

let emonde a=
	let liste_etats_utiles = etats_utiles a in
	let n = Array.length a.transitions in

	let sans_transition_inutile l=
		List.filter (fun (_,q) -> List.mem q liste_etats_utiles) l
	in

	for i=0 to n-1 do
		if not (List.mem i liste_etats_utiles) then
			a.transitions.(i) <- []
		else
			a.transitions.(i) <- sans_transition_inutile a.transitions.(i)
	done;
;;
\end{minted}

\section{Automate non déterministe}
\subsection{Principe}

Pour un état $q$ et une lettre $x$, on autorise plusieurs transition depuis $q$ étiquetées par $x$.
Donc pour un même mot m, il peut y avoir plusieurs chemin étiquetés par $m$. $m$ sera reconnu si au moins un de ces
chemin arrive dans $\mathcal{F}$\\

Ces automates sont beaucoup plus simple à concevoir et à créer que les AFD vus au début du cours. Le défaut est qu'il est
plus compliqué à programmer et plus lent

En effet à chaque lettre on doit calculer $\delta$ pour chaque élement d'une liste d'états au lieu d'un seul état.
Il est possible de déterminiser automatiquement n'importe quel AFND

\subsection{Définition précise}

\begin{definition}
    Un automate non deterministe sur $\Sigma$ est un quadruplet formé de
    \begin{itemize}
        \item Un ensemble $\mathcal{Q}$ appelé ensemble des états
        \item Une partie $\mathcal{I}$ de $\mathcal{Q}$ appelé ensemble des états initiaux
        \item Une partie de $\mathcal{Q}$ appelé ensemble des états finals
        \item Une fonction $\delta : \mathcal{Q}\times\Sigma \longrightarrow \mathcal{P}(\mathcal{Q})$ appelée fonction de transition 
    \end{itemize}
\end{definition}

\begin{remark}
    Pour $q\in \mathcal{Q},x\in\Sigma, \delta(q,x)$ peut être $\emptyset$. C'est l'analogue d'un blocage 
\end{remark}

\begin{definition}
    Soit $(\mathcal{Q},\mathcal{I},\mathcal{F},\delta)$ un AFND. On défini la fonction de transition étendue $\delta^*$ 
    par récursivité ainsi :
    \begin{itemize}
        \item $\forall q \in \mathcal{Q}, \delta^*(q,\epsilon)=\{q\}$
        \item $\forall q \in \mathcal{Q}, \forall m \in \Sigma^*, \forall x \in \Sigma, \delta^*(q,mx)=\cup_{r\in \delta^*(q,m)}\delta(r,x)$
    \end{itemize}
\end{definition}

\begin{definition}
    Soit A un AFDN et $(\mathcal{Q},\mathcal{I},\mathcal{F},\delta)$ ses composantes.\\
    $\forall m \in \Sigma^*$, on dit que $m$ est reconnu par $\mathcal{A}$ lorsque
    $\exists i \in \mathcal{I},\exists f \in \mathcal{F} \text{ tq } f\in \delta^*(i,m)$ ou
    $\cup_{i\in \mathcal{I}} \delta^*(i,m)\cap \mathcal{F} \not= \emptyset$

    On note $\mathcal{L}(\mathcal{A})$ l'ensemble des mots reconnus par $\mathcal{A}$
\end{definition}

Interpretation en terme de chemin : $\forall m \in \Sigma^*, m\in \mathcal{L}(\mathcal{A}) \Leftrightarrow$ il existe un chemin
étiqueté par m partant d'un état initial arrivant à un état final

\subsection{Programmation}

INSERT CODE TODO

\subsection{Determinisation}

Soit $(\mathcal{Q},\mathcal{I},\mathcal{F},\delta)$ un AFND noté $\mathcal{A}$.
On définit l'AFD $\mathcal{A}_d$ ainsi :\\

\begin{itemize}
    \item On pose $\mathcal{Q}_d=\mathcal{P}(\mathcal{Q})$. Ainsi un état de $\mathcal{A}_d$ est un ensemble d'états de $\mathcal{A}$. On appellera $\mathcal{A}_d$ l'automate des parties de $\mathcal{A}$
    \item On prend comme état initial $\mathcal{I}$
    \item On pose $\delta_d :^{\mathcal{Q}_d \times\Sigma \longrightarrow \mathcal{Q}_d}_{(X,x)\rightarrow \bigcup_{q\in X}\delta(q,x)}$
    \item On pose $\mathcal{F}_d = \{X\in\mathcal{P}(\mathcal{Q}) | X\cap F \not= \emptyset \}$
\end{itemize}

On pose alors $\mathcal{A}_d = (\mathcal{Q}_d,\mathcal{I},\mathcal{F}_d,\delta_d)$. C'est un AFD

Principe :  $\forall X \in \mathcal{P}(\mathcal{Q}),\forall m \in \Sigma^*,\delta^*_d(X,m)$ est l'ensemble des états accessibles
en lisant m depuis un élément de X.

\begin{proposition}
    $\forall X\in \mathcal{P}(\mathcal{Q}), \forall m \in \Sigma^*, \delta_d^*(X,m)= \bigcup_{r\in X}(r,m)$
\end{proposition}

\begin{dem}
    $\forall n \in \mathbb{N},$ soit $\mathcal{P}(n):"\forall  m \in \Sigma^*, \forall X \in \mathcal{P}(\mathcal{Q}), \delta^*_d(X,m)=\bigcup_{r\in X}(r,m)"$\\
    [INIT] : Soit $m \in \Sigma^0$ alors $m=\epsilon$. Alors $\delta^*_d(X,\epsilon)=X$ et $\bigcup_{r\in X}(r,\epsilon)=\bigcup_{r\in X} \{r\}=X$. Donc $\mathcal{P}(0)$
    [HERE] : Soit $n\in \mathbb{N} tq \mathcal{P}(n)$. Soit $m \in \Sigma^{n+1}$ et $X \in \mathcal{P}(\mathcal{Q})$.
    Soit $m' \in \Sigma^{n}$ et $x \in \Sigma tq m = m'x$
    On a $\delta^*_d(X,m)=\delta^*_d(X,m'x) = \delta_d(\delta^*_d(X,m'),x)= \bigcup_{q\in \bigcup_{r\in X} \delta^*(r,m')}\delta(q,x)= \bigcup_{r\in X}\bigcup_{q\in \delta^*(r,m')} \delta(q,x)= \bigcup_{r\in X} \delta^*(r,m'x)$
\end{dem}

\begin{theorem}
    $\mathcal{L}(\mathcal{A}_d)=\mathcal{L}(\mathcal{A})$
\end{theorem}

\begin{corollary}
    Un langage est reconnaissable par un AFND ssi il est reconnaissable par un AFD
\end{corollary}

\begin{dem}
    Soit $m\in \Sigma^*$. $m\in\mathcal{L}(\mathcal{A})\Leftrightarrow \exists q \in \mathcal{I} tq \delta^*(q,m)\cap\mathcal{F}\not=\emptyset \Leftrightarrow \bigcup_{q\in \mathcal{I}}\delta^*(q,m)\cap\mathcal{F}\not=\emptyset \Leftrightarrow \delta^*(q,m)\in\mathcal{F}  $
\end{dem}

\begin{example}
    Soit l'automate exemple de la partie précedente qui reconnait "ici".\\
    Calculons $\mathcal{A}_d$. On prend $\mathcal{Q}_d=\mathcal{P}([[0,3]])$
    On va dessiner uniquement les états accessibles
    \begin{itemize}
        \item état initial : $\{0\}$
        \item états finals : tous les ensembles qui contiennent 3 (il y en a 8)
        \item On va faire un tableau et le remplir ligne après ligne
    \end{itemize}

METTRE LE TABLEAU TODO
\end{example}

\subsection{Programmation de la determination}
On a déjà $\delta_d$. La seule difficulté, en Ocaml, un état est representé par un entier. Il faut attribuer un entier à chaque élément de $\mathcal{Q}_d$
On va numéroter uniquement les états accessible par le même algo que l'exemple precédent 

Etape 1 : numéroter les états accessibles dans $\mathcal{A}_d$. Les états seront représentées par des int list strictements croissantes.
On va de plus numéroter ces états grace à des dictionnaire (int list -> int). On enregistrera aussi un tableau pour associer à chaque numéro l'état correspondant.

\section{Expressions régulières}

\begin{definition}
    On définit les expressions regulieres sur $\Sigma$ :
    \begin{itemize}
        \item $\emptyset \epsilon$ et les lettres sont des regex
        \item Si $e$ est une regex, alors $(e^x)$ aussi
        \item Si $e$, $f$ sont deux regex, alors $(e+f)$ et $(e\cdot f)$ aussi
    \end{itemize}
\end{definition}

On définit le type Ocaml par :

TODO mettre le type ocaml

\begin{definition}
    Notons $\mathcal{R}(\Sigma)$ l'ensemble des regex sur $\Sigma$
    On définit la fonction $\mathcal{L}:\mathcal{R}(\Sigma)\rightarrow\mathcal{P}(\Sigma^*)$ par recursivité :
    \begin{itemize}
        \item $\mathcal{L}(\emptyset)=\emptyset$
        \item $\mathcal{L}(\epsilon)=\{\epsilon\}$
        \item $\forall x \in \Sigma, \mathcal{L}(x)=\{x\}$
        \item $\forall e \in \mathcal{R}(\Sigma),\mathcal{L}(e*)=\mathcal{L}(e)*$
        \item $\forall (e,f)\in \mathcal{R}(\Sigma), \mathcal{L}(e+f)=\mathcal{L}(e)\cup \mathcal{L}(f)$
        \item $\forall (e,f)\in \mathcal{R}(\Sigma), \mathcal{L}(e\cdot f)=\mathcal{L}(e)\cdot \mathcal{L}(f)$
    \end{itemize}
\end{definition}

\begin{proposition}
    Cette propriété permet de se passer des constructeurs $\emptyset$ et $\epsilon$\\
    Soit $e$ une regexp ni $\emptyset$ ni $\epsilon$, alors il existe $e'$ une regexp ne contenant ni $\emptyset$ ni $\epsilon$ tq $\mathcal{L}(e) = \mathcal{L}(e')$
\end{proposition}

\begin{proof}
    Par induction structurelle : \\

    \begin{itemize}
        \item Soit $e$ un lettre : $e$ ne contient ni $\emptyset$ ni $\epsilon$ alors $e'=e$ convient
        \item Soit $f\in \mathcal{R}(\Sigma)$ et $e=f$. Supposons la prop vraie pour f. Soit $f'$ une regexp sans $\emptyset$ ni $\epsilon$. Alors $\mathcal{L}(e) = \mathcal{L}(f^*)=\mathcal{L}(f)^*=\mathcal{L}(f)^* ou (\mathcal{L}(f')\cup\{\epsilon\})=\mathcal{L}(f')^*$.\\ Alors on prend $e'=f'*$
    TODO ajouter la fin de la demo
    \end{itemize}
\end{proof}

En consequence de cette proposition, on considerera dans la suite des regexp construits sans $\epsilon$ ni $\emptyset$.\\

\section{Langages Locaux}
\subsection{Programmation}

TODO AJOUTER CODE

\subsection{Automate associé à un langage}

\begin{remark}
    Si $\mathcal{L}(e)$ n'est pas local, on peut quand même calculer les ensembles. Dans ce cas on aura $\mathcal{L}(e) \not= \mathcal{L}(\mathcal{P},\mathcal{S},\mathcal{F},\alpha)$
\end{remark}

Soit $\mathcal{L}$ un langage local et $(\mathcal{P},\mathcal{S},\mathcal{F},\alpha)$ ses paramètres

On créer un état par lettre et un état qui servira d'état initial.\\

$\forall x \in \Sigma$, notons $q_x$ l'état associé à $x$ et notons $q_0$ l'état supplémentaire.\\
Soit $\mathcal{Q}=\{q_0\}\cup\{q_x;x\in \Sigma\}$\\

- Etat initial : $q_0$\\

- Transitions : 
\begin{itemize}
    \item $\forall a,b \in \mathcal{F}$, on ajoute la trans $q_a \rightarrow b \rightarrow q_b$
    \item $\forall x \in \mathcal{P}$, on ajoute la trans $q_0 \rightarrow x \rightarrow q_x$
\end{itemize}

- Etats finals :
\begin{itemize}
    \item On prend $q_0$ ssi $\alpha$
    \item Et on prend les $\{q_s ; s\in \mathcal{S} \}$
\end{itemize}

\begin{exemple}
    Soit $e=a^*(b+c)^*d$
    On a    $\mathcal{P} = \{a,b,c,d\}$
            $\mathcal{S} = \{d\}$
            $\alpha=\bot$
            $\mathcal{F} = \{aa,ab,ac,bb,cc,bc,cb,bd,cd,ad\}$

    On consate que cet automate reconnait $\mathcal{L}(e)$. De plus, il est local
\end{exemple}

\begin{theorem}
    Si $\mathcal{L}$ est local, $\mathcal{A}$ reconnait $\mathcal{L}$
\end{theorem}

\begin{proof}
    Soit $m \in \mathcal{L}$\\
    \begin{itemize}
        \item Si $m=\epsilon$ alors $\alpha$ donc $q_0$ est final. Or $\delta^*(q_0,epsilon)=q_0$ donc $\epsilon \in \mathcal{L}$
        \item Si $m\not= \epsilon$, on a $m=m_0...m_{n-1}$. \begin{itemize}
            \item $m_0 \in \mathcal{P}$ donc on a la transition $q_0 \rightarrow m_0 \rightarrow q_{m_0}$
            \item $\forall i \in [[0,n-1[[, m_im_{i+1} \in \mathcal{F}$ \footnote{Ce n'est pas l'ensemble des états finaux}
        \end{itemize}
    \end{itemize}
    Ainsi, $\mathcal{A}$ contient le chemin.\\
    Donc $m\in \mathcal{L}(\mathcal{A})$\\

    Soit $m\in\mathcal{L}(\mathcal{A})$
    \begin{itemize}
        \item Si $m=\epsilon$, $q_0$ est final, donc $\alpha = \top$ donc $\epsilon\in\mathcal{L}$
        \item Sinon : $m=m_0...m_{n-1}$ \begin{itemize}
            \item Pas de blocage en lisant $m_0$ donc il existe une transtion $q_0 \rightarrow m_0 \rightarrow q_{m_0}$ Donc $m_0\in \mathcal{P}$
            \item $\forall i \in [[0,n-1[[$, apres lecture de $m$ on est dans $q_{m_i}$ et il n'y a pas de blocage ensuite. Alors la transition  $q_{m_1} \rightarrow m_i \rightarrow q_{m_{i+1}}$
        \end{itemize}
        \item apres lecture de $m$ on est dans $q_{m_{n-1}}$ Or cet état est final car $m$ est reconnu
    \end{itemize}
    Ainsi $m\in\mathcal{L}(\mathcal{P},\mathcal{S},\mathcal{F},\alpha)$
\end{proof}



\end{document}






