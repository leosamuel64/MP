\begin{document}

Il s'agit d'une variante de la récurrence.

\section{Récurrence classique sur $\mathbb N$}

$\mathbb{N}$ peut etre définis par 

\begin{minted}[
    frame=lines,
    framesep=2mm,
    baselinestretch=1.2,
    bgcolor=white,
    fontsize=\footnotesize,
    linenos]
    {ocaml}
type entier_nat = 0 | succ of entier_nat;; 
\end{minted}

\begin{th}
    Théorème de récurrence : \\
    On suppose :
    \begin{itemize}
        \item $P(0)$
        \item $\forall n \in \mathbb{N}, P(n)\Rightarrow P(n+1)$
    \end{itemize}
    Alors $\forall n \in \mathbb{N}, P(n)$
\end{th}

\section{Pour les formules}

Le type des formules définis à l'aide de 3 constructeurs récursif (Ou,Et,Non) et 2 constructeurs non récursifs (Constante,Variable) \\
Le théorème devient alors :
\begin{th} \\
    Soit $P$ un prédicat sur $\mathbb{F}$\\
    On suppose : \\
    \begin{itemize}
        \item $\forall b \in B, P(Constante b)$
        \item $\forall x \in \Sigma, P(Variable x)$
        \item $\forall f \in \mathbb{F}, P(f)\Rightarrow P(Non f)$
        \item $\forall f_1,f_2 \in \mathbb{F}, P(f_1)\land P(f_2)\Rightarrow P(f_1\land f_2)$
        \item $\forall f_1,f_2 \in \mathbb{F}, P(f_1)\land P(f_2)\Rightarrow P(f_1 \lor f_2)$
    \end{itemize}
    Alors $\forall f \in \mathbb{F}, P(f)$
\end{th}

\begin{rq}
    Les deux premiers points correspondent à l'initialisation et les trois derniers à l'hérédité
\end{rq}

\begin{ex}
    Exercice 9
\end{ex}

\begin{dem}
    Il faut faire une récurrence forte sur la hauteur de la formule
    \begin{itemize}
        \item Les deux premiers points prouvent que $P$ est vrai pour les formules de hauteur $0$
        \item Les trois points suivant prouvent que $\forall n \in \mathbb{N}$, si $P$ est vrai pour toute formule de hauteur inférieur à $n$
    \end{itemize}
    alors $P$ est vrai pour toute formule de hauteur n+1
\end{dem}

\begin{rq}
    \begin{itemize}
        \item Une preuve par induction structurelle peut toujours être remplacée par une récurrence classique sur la hauteur
        \item Le concept est le même que pour les arbres.
    \end{itemize}
\end{rq}

\begin{ex} Exemple de rédaction : Exercice 9 \\
    DEMAIN
\end{ex}





































    
\end{document}

























