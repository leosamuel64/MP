\begin{document}
\maketitle
Il s'agit d'une variante de la récurrence.

\section{Récurrence classique sur les entiers}

$\mathbb{N}$ peut etre définis par 

\begin{minted}[
    frame=lines,
    framesep=2mm,
    baselinestretch=1.2,
    bgcolor=white,
    fontsize=\footnotesize,
    linenos]
    {ocaml}
type entier_nat = 0 | succ of entier_nat;; 
\end{minted}

\begin{thr}
    Théorème de récurrence : \\
    On suppose :
    \begin{itemize}
        \item $P(0)$
        \item $\forall n \in \mathbb{N}, P(n)\Rightarrow P(n+1)$
    \end{itemize}
    Alors $\forall n \in \mathbb{N}, P(n)$
\end{thr}

\section{Pour les formules}

Le type des formules définis à l'aide de 3 constructeurs récursif (Ou,Et,Non) et 2 constructeurs non récursifs (Constante,Variable) \\
Le théorème devient alors :
\begin{thr} \\
    Soit $P$ un prédicat sur $\mathcal{F}$\\
    On suppose : \\
    \begin{itemize}
        \item $\forall b \in B, P(\text{Constante } b)$
        \item $\forall x \in \Sigma, P(\text{Variable } x)$
        \item $\forall f \in \mathcal{F}, P(f)\Rightarrow P(\lnot f)$
        \item $\forall f_1,f_2 \in \mathcal{F}, P(f_1)\land P(f_2)\Rightarrow P(f_1\land f_2)$
        \item $\forall f_1,f_2 \in \mathcal{F}, P(f_1)\land P(f_2)\Rightarrow P(f_1 \lor f_2)$
    \end{itemize}
    Alors $\forall f \in \mathcal{F}, P(f)$
\end{thr}

\begin{rmq}
    Les deux premiers points correspondent à l'initialisation et les trois derniers à l'hérédité
\end{rmq}

\begin{dem}
    Il faut faire une récurrence forte sur la hauteur de la formule
    \begin{itemize}
        \item Les deux premiers points prouvent que $P$ est vrai pour les formules de hauteur $0$
        \item Les trois points suivant prouvent que $\forall n \in \mathbb{N}$, si $P$ est vrai pour toute formule de hauteur inférieur à $n$
    \end{itemize}
    alors $P$ est vrai pour toute formule de hauteur n+1
\end{dem}

\begin{rmq} Structure\\ 
    \begin{itemize} \\
        \item Une preuve par induction structurelle peut toujours être remplacée par une récurrence classique sur la hauteur
        \item Le concept est le même que pour les arbres.
    \end{itemize}
\end{rmq}
\newpage

\begin{ex} Exemple de rédaction : Exercice 9 \\
    $\forall f \in \mathcal{F}$, notons $P(f)=$"f est équivalente à une formule n'utilisant que des constantes, variables et des $\overline{\land}$ \\
    
    Initialisation : On traite les constructeurs non recursifs\\
        \begin{itemize}
            \item Soit $f$ une variable, $f \equiv f$
            \item Soit $f$ une constante, $f\equiv f$
        \end{itemize}
    \newline
    Hérédité : On traite les constructeurs récursifs\\
    \newline
        Soit $f \in \mathcal{F}$ tq $P(f)$ \\
        
        Montrons $P(\lnot f)$ : Par $P(f), \exists f' \in \mathcal{F}, f\equiv f'$ alors $\lnot f \equiv f' \overline{\land} f'$ \\
        \newline
        D'après $P(f),\exists f',g' \in \mathcal{F} \text{ tq } f'\equiv f, g'\equiv g$\\
        \begin{itemize}
            \item $f \land g \equiv (f' \overline{\land} g')\overline{\land}(f' \overline{\land} g')$
            \item $f \lor g \equiv (f' \overline{\land} f')\overline{\land}(g' \overline{\land} g')$
        \end{itemize}

Alors  $\forall f \in \mathcal F, f$ est equivalente à une formule n'utilisant que des constantes, des variables et des $\overline{\land}$

\end{ex}

    
\end{document}
