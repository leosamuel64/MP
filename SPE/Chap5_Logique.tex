\begin{document}
\maketitle

\section{Calcul booléen}

\subsection{Notations}

\begin{itemize}
    \item Vrai : $\top$
    \item Faux : $\bot$
    \item Non : $\lnot$
    \item Et : $\land$
    \item Ou : $\lor$
\end{itemize}
\newline

De plus, dans ce chapitre, on notera $B=\{\top,\bot\}$ \\
Ordre de priorité dans les calculs : $\lnot$,$\land$,$\lor$ \\

\subsection{Règles de calcul}

\begin{prop}
    $\land$ :
    \begin{itemize}
        \item Commutatif
        \item Associatif
        \item Neutre : $\top$
    \end{itemize}
\end{prop}

\begin{prop}
    $\lor$ :
    \begin{itemize}
        \item Commutatif
        \item Associatif
        \item Neutre : $\bot$
    \end{itemize}
\end{prop}

\begin{prop}
    Entre $\lor$, $\land$ et $\lnot$
    \begin{itemize}
        \item Distributivité ($\lor$ est distributif sur $\land$ et inversement contrairement aux nombres)
        \item Loi de De Morgan : $\forall (a,b,c) \in B^3$
            \begin{itemize}
                \item $\lnot(a\land b)= \lnot a \lor \lnot b$
                \item $\lnot(a\lor b)= \lnot a \land \lnot b$
            \end{itemize}
    \end{itemize}
\end{prop}

\begin{dem}
    Montrons que $\forall (a,b,c) \in B^3, a\land(b\lor c)=(a\land b)\lor (a \land c)$ \\
    On étudie toutes les possibilités pour ($(a,b,c))$. Il y en a $8$ car $2^3=8$ \\
    On les regroupe dans une table de vérité \\

        \begin{tabular}{|l|l|l|l|l|l|l|l|}
        \hline
        $a$ & $b$ & $c$ & $b\lor c$ & $a\land (b\lor c$ & $a\land b$ & $a \land c$ & $(a\land b)\lor (a\land c)$ \\ \hline
        0   & 0   & 0   & 0         & 0                 & 0          & 0           & 0                           \\ \hline
        0   & 0   & 1   & 1         & 0                 & 0          & 0           & 0                           \\ \hline
        0   & 1   & 0   & 1         & 0                 & 0          & 0           & 0                           \\ \hline
        0   & 1   & 1   & 1         & 0                 & 0          & 0           & 0                           \\ \hline
        1   & 0   & 0   & 0         & 0                 & 0          & 0           & 0                           \\ \hline
        1   & 0   & 1   & 1         & 1                 & 0          & 1           & 1                           \\ \hline
        1   & 1   & 0   & 1         & 1                 & 1          & 0           & 1                           \\ \hline
        1   & 1   & 1   & 1         & 1                 & 1          & 1           & 1                            \\ \hline
        \end{tabular}


    Les colonnes sont identiques \\
\end{dem}

Les propriétés de $\land$ et $\lor$ ressemblent à celle pour $\cdot$ et $+$ sur $\mathbb{Z}/n\mathbb{Z}$ \\

On utilise alors souvent : 
\begin{itemize}
    \item $0$
    \item $1$
    \item $\cdot$
    \item $+$
\end{itemize}

Dans ce cas, les loi de DE Morgan sont :
\item Loi de De Morgan : $\forall (a,b,c) \in B^3$
\begin{itemize}
    \item $\overline{(a\land b)}= \overline a + \overline b$
    \item $\overline{(a + b)}= \overline a  \overline b$
\end{itemize}

Attention dans ce cas, $1+1=1$

\begin{rq}
    $(\mathbb{Z}/n\mathbb{Z},+,\cdot)$ est un corps. \\
    $(B,\lor,\land)$ n'est même pas un anneau car $1$ n'a pas d'opposé
\end{rq}

\begin{rq}
    Notons $\lxor$ le ou exclusif : \\
    $1\lxor 1=0$
    On peut vérifier que $(B,\lxor,\land)$ est un corps , isomorphe à $\mathbb{Z}/n\mathbb{Z}$
\end{rq}

\subsection{Autres connecteurs logiques}
Combien y-a-t-il de fonction de $B$ dans $B$ ? \\
Il y en a $|B|^{|B|}=4$ \\
Il y a $id,\lnot,0,1$ \\

Combien y-a-t-il de fonction de $B^2$ dans $B$ ?
Il y en a 16. Nous connaisons déja $\lor,\land,\lxor$

Pour definir un opérateur, il suffit de donnée sa table de vérité

\subsubsection{L'implication}
\begin{def}
    $\forall a,b \in B$, on note $a\Rightarrowb$ le booléen $b\lor\lnot a$
\end{def}

\begin{ex}
    "Mange ta soupe ou va dans ta chambre"  by FP\\
    donne \\
    "Si tu ne mange pas ta soupe alors va dans ta chambre"\\
\end{ex}

Interpretation en mathématiques :\\
Table de vérité de $\Rightarrow$ :

    \begin{tabular}{|l|l|l|}
    $a$ & $b$ & $a\Rightarrow b$ \\ \hline
    0   & 0   & 1                \\ \hline
    0   & 1   & 1                \\ \hline
    1   & 0   & 0                \\ \hline
    1   & 1   & 1     \\ \hline          
    \end{tabular}


Ainsi, la formule $\forall (a,b) in B^2, a \Rightarrow b$ est vraie si et seulement si a chaque fois que a est vrai, b l'est aussi. (Quand a est faux, b peut valoir n'importe quoi !)

Digressions : \\
Un théorème est une formule $\forall x \in E, P(x) \Rightarrow Q(x)$ où $E$ est un ensemble et $P,Q$ des prédicats sur $E$. (fonctions de $E$ dans $B$)
Dans le cas où $P$ est faux, on a aucune informations sur $Q$. 

Pour montrer q'un théorème est faux, il faut montrer
$\lnot(\forall x \in E, P(x) \Rightarrow Q(x))$ cad $\exists x \in E, \lnot(P(x) \Rightarrow Q(x))$ cad $\exists x \in E, \lnot Q(x) \land P(x)$
Ainsi, cela revient à trouver un $x\in E$ pour lequel l'hypothèse est vraie mais la conclusion est fausse. Un tel $x$ s'appelle un contre-exemple

\subsubsection{Equivalents}
\begin{def}
    $\forall (a,b) \in B^2$, on note $a<\Rightarrowb$ le booléen $(a\Rightarrowb) \land (b\Rightarrowa)$
    C'est en réalité la même chose que $=$ pour les booléens
\end{def}

\subsubsection{Non-et}
\begin{def}
    $\forall (a,b) \in B^2,\lnot(a\land b)$ 
\end{def}

Il est utilisé car il coute seulement 3 transistors. \\
On peut definir les autres opérations à partir du non-et. (cf Exercice 9) \\

De même on définit non-ou.\\

En général dans un langage, il est seulement fournit $\lnot,\land,\lor$ desquels on peut définir les autres.


\section{Formules logiques}

\subsection{Type Caml}

Voici un type pour représenter les formules logiques :

\begin{minted}[
    frame=lines,
    framesep=2mm,
    baselinestretch=1.2,
    bgcolor=white,
    fontsize=\footnotesize,
    linenos]
    {ocaml}
    type formule =
        | Variable of string
        | Non of formule
        | Et of (formule*formule)
        | Ou of (formule*formule)
        | Constante of bool
        ;;

    \end{minted}

    Ainsi l'exemple $\overline{a+b}\cdot c$ par :



\begin{minted}[
    frame=lines,
    framesep=2mm,
    baselinestretch=1.2,
    bgcolor=white,
    fontsize=\footnotesize,
    linenos]
    {ocaml}
    let exemple 1 = Et(Non(Ou(Variable "a",Variable "b")),Variable "c");;
    \end{minted}

Ce type est similaire à un type arbre.

\subsection{Définition Mathématique}

On définit le concept de formule abstraite.\\
Soit $\Sigma$ un ensemble dont les éléments seront appelés des "variable (booléenes)". On définit alors l'ensemble des formules logiques ainsi :\\

\begin{itemize}
    \item $\forall x \in \Sigma,x$ est une formule
    \item $\forall b \in B,b$ est une formule
    \item $\forall \text{ formule } f, \overline{f}$ est une formule
    \item $\forall \text{ formules } f_1,f_2, f_1+f_2 \text{ et } f_1f_2$ sont des formules
\end{itemize}

\begin{exemple}\\
    Soit $a,b,c \in B$\\
    $\overline{(a+b)}c$ est une formule\\
    $\lor \lor a$ n'est pas une formule\\
\end{exemple} 

\subsection{Sémantique}

Donnons un sens aux formules abstraites définies ci-dessus.\\

En pratique, on va définir des fonctions qui utilisent ces formules

\begin{def}
    Un contexte (ou distribution de vérité ou valuation) sur $\Sigma$ est une fonction de $\Sigma$ dans $B$.\\
\end{def}

Choisir un contexte sur $\Sigma$ revient à choisir la valeur de chaque variable.

\begin{def}
    Soit $\phi$ un contexte sur $\Sigma$. Notons $\mathbb{F}$ l'ensemble des formule sur $\Sigma$.\\

    On définit la fonction "Evaluation en $\phi$", qu'on note $E_v_{\phi}$\\
    Ainsi:\\
    $\forall b \ in B, E_v_{\phi}(b)=b$ \\
    $\forall x \ in \Sigma, E_v_{\phi}(x)=\phi(x)$ \\
    $\forall f in \mathbb{F},E_v_{\phi}(\lnot f)=\lnot E_v_{\phi}(f)$ \\
    $\forall f_1,f_2 in \mathbb{F},E_v_{\phi}(f_1f_2)= E_v_{\phi}(f_1)E_v_{\phi}(f_2)$ \\
    $\forall f_1,f_2 in \mathbb{F},E_v_{\phi}(f_1+f_2)= E_v_{\phi}(f_1)+E_v_{\phi}(f_2)$ \\
\end{def}

Pour représenter un contexte en Ocaml, utilisons un dictionnaire. Pour des exemples avec peu de variables, utilisons les listes d'association.\\

\begin{minted}[
    frame=lines,
    framesep=2mm,
    baselinestretch=1.2,
    bgcolor=white,
    fontsize=\footnotesize,
    linenos]
    {ocaml}
    type contexte = (string * bool) list;;
    \end{minted}

La fonction d'évaluation devient alors : \\ 

\begin{minted}[
    frame=lines,
    framesep=2mm,
    baselinestretch=1.2,
    bgcolor=white,
    fontsize=\footnotesize,
    linenos]
    {ocaml}
    let rec evaluation contexte formule=
        match formule with
        | Constante b -> b
        | Variable x -> List.assoc x contexte
        | Non f ->not (evaluation contexte f)
        | Ou (f1,f2) -> (evaluation contexte f1) || (evaluation contexte f2)
        | Et (f1,f2) -> (evaluation contexte f1) && (evaluation contexte f2)
    ;;
    \end{minted}

\subsection{Vocabualire}

Notons toujours $\mathbb{F}$ l'ensemble des formules. Notons $\mathbb{C} l'ensemble des contextes sur $\Sigma$

\begin{def}
    Soit $f \in \mathbb{F}$
    \begin{itemize}
        \item On dit que $f$ est satisfiable lorsque $\exists \phi \in \mathbb{C} tq E_v_{\phi}(f)=\top$
        \item On dit que $f$ est tautologique lorsque $\forall \phi \in \mathbb{C} tq E_v_{\phi}(f)=\top$
        \item Soit $g \in mathbb{F}$. On dit que $f$ est équivalente à $g$ ($f\equiv g$) lorsque $E_v_{\phi}(f)=E_v_{\phi}(g)$
    \end{itemize}
\end{def}

Ecrivons une fonction qui teste si une formule est satisfiable :\\

Méthode naive : calculons la liste de tous les contextes :




















\end{document}