\begin{document}

\section{Calcul booléen}

\subsection{Notations}

\begin{itemize}
    \item Vrai : $\true$
    \item Faux : $\false$
    \item Non : $\lnot$
    \item Et : $\land$
    \item Ou : $\lor$
\end{itemize}

De plus, dans ce chapitre, on notera $B=\{\true,\false\}$ \\
Ordre de priorité dans les calculs : $\lnot$,$\land$,$\lor$ \\

\subsection{Règles de calcul}

\begin{prop}
    $\land$ :
    \begin{itemize}
        \item Commutatif
        \item Associatif
        \item Neutre : $\true$
    \end{itemize}
\end{prop}

\begin{prop}
    $\lor$ :
    \begin{itemize}
        \item Commutatif
        \item Associatif
        \item Neutre : $\false$
    \end{itemize}
\end{prop}

\begin{prop}
    Entre $\lor$, $\land$ et $\lnot$
    \begin{itemize}
        \item Distributivité ($\lor$ est distributif sur $\land$ et inversement contrairement aux nombres)
        \item Loi de De Morgan : $\forall (a,b,c) \in B^3$
            \begin{itemize}
                \item $\lnot(a\land b)= \lnot a \lor \lnot b$
                \item $lnot(a\lor b)= \lnot a \land \lnot b$
            \end{itemize}
    \end{itemize}
\end{prop}

\begin{dem}
    Montrons que $\forall (a,b,c) \in B^3, a\land(b\lor c)=(a\land b)\lor (a \land c)$ \\
    On étudie toutes les possibilités pour ($(a,b,c))$. Il y en a $8$ car $2^3=8$ \\
    On les regroupe dans une table de vérité \\

    --Insere Table de vérité-- \\

    Les colonnes sont identiques \\
\end{dem}

Les propriétés de $\land$ et $\lor$ ressemblent à celle pour $\cdot$ et $+$ sur $\mathbb{Z}/n\mathbb{Z}$ \\

On utilise alors souvent : 
\begin{itemize}
    \item $0$
    \item $1$
    \item $\cdot$
    \item $+$
\end{itemize}

Dans ce cas, les loi de DE Morgan sont :
\item Loi de De Morgan : $\forall (a,b,c) \in B^3$
\begin{itemize}
    \item $\barre{(a\land b)}= \barre a + \barre b$
    \item $\barre{(a + b)}= \barre a  \barre b$
\end{itemize}

Attention dans ce cas, $1+1=1$

\begin{rq}
    $(\mathbb{Z}/n\mathbb{Z},+,\cdot)$ est un corps. \\
    $(B,\lor,\land)$ n'est même pas un anneau car $1$ n'a pas d'opposé
\end{rq}

\begin{rq}
    Notons $\lxor$ le ou exclusif : \\
    $1\lxor 1=0$
    On peut vérifier que $(B,\lxor,\land)$ est un corps , isomorphe à $\mathbb{Z}/n\mathbb{Z}$
\end{rq}

\subsection{Autres connecteurs logiques}
Combien y-a-t-il de fonction de $B$ dans $B$ ? \\
Il y en a $|B|^{|B|}=4$ \\
Il y a $id,\lnot,0,1$ \\

Combien y-a-t-il de fonction de $B^2$ dans $B$ ?
Il y en a 16. Nous connaisons déja $\lor,\land,\lxor$

Pour definir un opérateur, il suffit de donnée sa table de vérité

\subsubsection{L'implication}
\begin{def}
    $\forall a,b \in B$, on note $a=>b$ le booléen $b\lor\lnot a$
\end{def}

\begin{ex}
    "Mange ta soupe ou va dans ta chambre"  by FP\\
    donne \\
    "Si tu ne mange pas ta soupe alors va dans ta chambre"\\
\end{ex}

Interpretation en mathématiques :\\
Table de vérité de $=>$
Ainsi, la formule $\forall (a,b) in \B^2, a => b$ est vraie si et seulement si a chaque fois que a est vrai, b l'est aussi. (Quand a est faux, b peut valoir n'importe quoi !)

Digressions : \\
Un théorème est une formule $\forall x \in E, P(x) => Q(x)$ où $E$ est un ensemble et $P,Q$ des prédicats sur $E$. (fonctions de $E$ dans $B$)
Dans le cas où $P$ est faux, on a aucune informations sur $Q$. 

Pour montrer q'un théorème est faux, il faut montrer
$\lnot(\forall x \in E, P(x) => Q(x))$ cad $\exists x \in E, \lnot(P(x) => Q(x))$ cad $\exists x \in E, \lnot Q(x) \land P(x)$
Ainsi, cela revient à trouver un $x\in E$ pour lequel l'hypothèse est vraie mais la conclusion est fausse. Un tel $x$ s'appelle un contre-exemple

\subsubsection{Equivalents}
\begin{def}
    $\forall (a,b) \in B^2$, on note $a<=>b$ le booléen $(a=>b) \land (b=>a)$
    C'est en réalité la même chose que $=$ pour les booléens
\end{def}

\subsubsection{Non-et}
\begin{def}
    $\forall (a,b) \in B^2,\lnot(a\land b)$ 
\end{def}

Il est utilisé car il coute seulement 3 transistors. \\
On peut definir les autres opérations à partir du non-et. (cf Exercice 9) \\

De même on définit non-ou.\\

En général dans un langage, il est seulement fournit $\lnot,\land,\lor$ desquels on peut définir les autres.

POUR JEUDI : Exercice 6

\section{Formules logiques}












\end{document}